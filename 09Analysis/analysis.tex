\section{Analysis}\label{sec:analysis}
% 

This section will go through how we applied the findings throughout the report and the discussion made while finding a solution to making a platooning system. First will be motivation and reasons for using a technology like platooning, after is how to build the system from a high level perspective and also from a lower level. Lastly is how the platooning system could work from the drivers perspective and what benefits/drawbacks it gives them.  
% 
\subsection{Motivation}
Companies internally can easily manage to arrange platooning by scheduling time and routes accordingly, to match with other deliveries. This can be done by having multiple trucks go from A to B on the same time or having their starting point in different location and then meet up at an intersection. Since it is inside the company, making the drivers use the platooning system to save fuel is a possible requirement to force on the drivers. This would lead to the company saving money through usage of fuel (ref) on the cost of more logistic and initial cost of platooning technology.\par
% 
This is slightly harder to do when it comes across different companies. The temporary solution could be that companies have a partnership, giving them the information needed to arrange their routes together to form platooning. This solution would be the same as the internal company, since it would be like 1 big company working together. Since the front truck uses more fuel than the later trucks, an agreement in the contract to share the fuel consumption would have to be made in the partnership. This however would only be a temporary solution since partnership would not be possible to do with everyone, and new upstarting companies or smaller companies would not be able to join in those partnerships to begin with.\par
% 
To take it one step further and remove the partnership and let anyone with the technology and wants to be in a platooning can do so. While internal companies can keep doing their platooning as usual, other nearby trucks can join the platoon on the cost of a small fee to the front truck for each km driven together in the platoon. This would be a solution where everyone benefits, since the initial company who started the platooning is saving even more fuel with the addition of the new truck giving them extra money from the fee. If another truck joins the platoon, the 2nd last truck pays a slightly higher fee than before (since their fuel consumption is the lowest of them all) and the last truck pays a small fee as usual\footnotemark. By giving a small fee to the fist truck, it will not discourage truck drivers to be the frontal truck and no matter where you are in the platoon you will always save money, even when being the last.
% 
(picture that explains the situation maybe?)
% 
\footnotetext{\url{http://www.greencarcongress.com/2007/07/platooning-redu.html}, accessed 18/05/2017}
% 
\subsection{Architecture}
%
Throughout the research, it has been discovered that barely any V2V communication is needed to solve the 3 requirements mentioned in the requirements section. One of our interviewee said that in one of the projects he was involved in, he was driving in a platoon formation from point A to point B with only the usage of the cruise control, leaving him to only be in control of steering. This makes the trucks use less fuel, because of the platooning formation and it improves the drivers experience, by releasing them from having to manage the speed and only take control of the steering. Since the sensors can react quicker than the human can, in case of a sudden break from the truck in front, the cruise control can break quicker leaving us with a higher safety. This basic platooning system however, does not cover the system requirements, and increases safety by a small amount compare to more advanced systems.\par
%
A more advanced platooning system would include V2V communication. This would give the trucks more information to calculate with. Instead of only relying on their own sensors, they can get information that any of the trucks in the platoon have access to, and use it for their own calculations to optimise the best outcome. Without the V2V communication, a platoon of 4 trucks, if the first truck breaks the last truck would start breaking as soon as the sensors realises that the 3rd truck is breaking. With V2V communication, as soon as the first truck breaks, it would give the message to all the trucks that it is breaking, giving the last truck more time to react to it. With V2V communication more information can be relayed to each truck, that otherwise might not be possible to do with sensors alone. An example of that would be the weight of the truck, giving the platoon reason to increase or decrease the distance of the trucks, if more space is needed in case of a sudden break happening. This can increase both safety and fuel consumption, since it can minimise the amount of space needed between the trucks, better than having a specific length between.\par
%
The full architecture of the system we suggest would be like the VANET system. Having both V2V and V2I communication, as well as a back-end system holding the information, making information flow much broader than the communication between only 2 entities. This also follows all our system requirements as well as improve the safety significantly. By using a back-end server that stores data from other trucks, any trucks in the system can use it and through process see what happened or is happening later down the route, the trucks can be more prepared for it, far before their own sensors reaches that point. This also allows for more user-friendly experience, with having a system that tracks all platooning happening, letting the driver know, where the nearest platoon is, in case they want to join one.
% 
\subsection{Technology}
% 
During the research we came across four possible technology solutions that would be capable of facilitating platooning in VANET-like fashion. These technologies represent the attempts for standardisation in Europe (ETSI ITS-G5), United States (WAVE), Japan (ARIB STD-T109) and world-wide (802.11). All of them address issues that are specific for vehicular environment. These issues include short window for communication among nodes due to high mobility, fading and interference related problems and timing challenges. Three of these technologies provide solution for these problems on PHY and MAC layer only, while one also includes solutions for higher layers. All of these technologies were specifically designed for vehicular use and the challenges mentioned above were kept in mind.\par
% 
While there is many possible scenarios how these technologies could be utilised, ranging from automated toll collection to forward collision warning, and platooning is only one of these scenarios, our research has shown that all of the technologies researched are capable of carrying any communication required to enable platooning. There is certain limitations, such as lower QoS during higher density of the nodes in one area, for example, and the research should be directed into resolving them. These problems, however, do not prevent the platooning from being enabled. After all, during our research we came across cases where platooning was successfully tried out and even case where platooning is used regularly. While we could not find the technology behind some of them, the majority of the cases were using one of the standards mentioned in this report.\par
% 
%
\subsection{Autonomous System}
Projects have given the truck drivers different level of control while platooning. Some had the front truck driver driving the truck manually or with a speed controller, while the following trucks only needed to steer, and having the speed and break be controlled by the computer. Others had speed, break steering being taken over by the computer. Both cases had a driver ready to take over in case something would go wrong and is in need of a overwriting of the truck driver.\par
% 
Around 94 percentage of car crashes in the US (2015) is estimated to be the result of the driver \cite{HighwayTrafficSafetyAdministration2015TRAFFICSurvey}.
This can be drastically reduced by having autonomous driver, meaning, giving the driver as little control as possible. This is on paper a good thing, but is a completely different thing in reality. As most things that concerns safety of human life, when new technology appears, many are in distrust regarding it. A statistic shows that in US and UK there is as little as 15 percentage who are not concerned with this technology at all, while 26 percentage is very concerned in the US \cite{Schoettle2014AAustralia}. 
