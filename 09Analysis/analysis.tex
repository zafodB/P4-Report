\section{Analysis}\label{sec:Analysis}

//Analysis goes here\par
% 
\subsection{Motivation}

Companies internally can easily manage to arrange platooning by scheduling time and routes accordingly, to match with other deliveries. This can be done by having multiple trucks go from A to B on the same time or having their starting point in different location and then meet up at an intersection. Since it is inside the company, making the drivers use the platooning system to save fuel is a possible requirement to force on the drivers. This would lead to the company saving money through usage of fuel (ref) on the cost of more logistic and initial cost of platooning technology.\par
% 
This is slightly harder to do when it comes across different companies. The temporary solution could be that companies have a partnership, giving them the information needed to arrange their routes together to form platooning. This solution would be the same as the internal company, since it would be like 1 big company working together. Since the front truck uses more fuel than the later trucks, an agreement in the contract to share the fuel consumption would have to be made in the partnership. This however would only be a temporary solution since partnership would not be possible to do with everyone, and new upstarting companies or smaller companies would not be able to join in those partnerships to begin with.\par
% 
To take it one step further and remove the partnership and let anyone with the technology and wants to be in a platooning can do so. While internal companies can keep doing their platooning as usual, other nearby trucks can join the platoon on the cost of a small fee to the front truck for each km driven together in the platoon. This would be a solution where everyone benefits, since the initial company who started the platooning is saving even more fuel with the addition of the new truck giving them extra money from the fee. If another truck joins the platoon, the 2nd last truck pays a slightly higher fee than before (since their fuel consumption is the lowest of them all) and the last truck pays a small fee as usual. By giving a small fee to the fist truck, it will not discourage truck drivers to be the frontal truck and no matter where you are in the platoon you will always save money, even when being the last.
(picture that explains the situation maybe?)
% 
\subsection{Control WIP title etc etc}
Projects have given the truck drivers different level of control while platooning. Some had the front truck driver driving the truck manually or with a speed controller, while the following trucks only needed to steer, and having the speed and break be controlled by the computer. Others had speed, break steering being taken over by the computer. Both cases had a driver ready to take over in case something would go wrong and is in need of a overwriting of the truck driver.\par
% 
\par
< something > 
\par
% 
Around 94 percentage of car crashes in the US (2015) is estimated to be the result of the driver. This can be drastically reduced by having autonomous driver, meaning, giving the driver as little control as possible. This is on paper a good thing, but is a completely different thing in reality. As most things that concerns safety of human life, when new technology appears, many are in distrust regarding it. A statistic shows that in US and UK there is as little as 15 percentage who are not concerned with this technology at all, while 26 percentage is very concerned in the US. 
