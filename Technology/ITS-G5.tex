\subsubsection{ETSI ITS-G5}
% http://www.etsi.org/deliver/etsi_en/302600_302699/302663/01.02.00_20/en_302663v010200a.pdf
ITS-G5 is a Dedicated Short Range Communication (DSRC) standard, running on the 5GHz frequency band based on 802.11p in Europe.
The spectrum allocated for DSRC is 5.875-5.905 GHz also called ITS-G5A. This allocated spectrum of 30 MHz is only for ITS. Of the 30 MHz only a third of that is saved for the Road safety as seen on the picture below.\par
% 
The protocol stack for ETSI ITS G5 uses IEEE 802.11p as the physical layer. By dividing channels into Control Channel (CCH) and Service Channel (SCH), packets transmit through the control channel do not have to compete with service channel. For Medium access layer, channel access and priority is done by Enhanced Distributed Coordinated Function (EDCA) (same as CSMA/CA?). The packets not only compete for channel access with other vehicles but also with packets internally in the same node. Packets are divided by 4 access categories, VO, VI, BE and BK. Each access categories can be given one of the two channel types, resulting in 8 different combinations. Internal congestion control is being done (one for CCH and one for SCH) before the packet is sent to their respective queue. ETSI ITS G5 also performs a Decentralised Congestion Control (DCC), which changes transmit power, the minimum packet interval, the data rate, and the sensitivity of the radio accordingly. It does this by acting like a state machine going from a “active”, “relaxed” and “restrictive” state depending on the situation.
ETSI ITS G5 periodically transmit broadcast message called CAMs with information about their current state, location, speed, and direction with a frequency of 10 Hz.\par
% 
As described in conference proceedings for IEEE 79th Vehicular Technology Conference: \textquote[\cite{Shi2014SpectrumSafety}]{\textit{We performed extensive simulation for CSMA/CA and STDMA MAC (Simulation parameters) schemes in an urban highway scenario with realistic traffic density. Results show that more than 80 MHz is required to achieve 1\% packet loss with 500 m communication range. It is significantly larger than the current spectrum allocated of 10 MHz in the US and Europe.}}\par 
% 
It is also mentioned that by decreasing the communication range to 100 m, the spectrum requirement is reduced to 20 MHz, still being twice the amount available today. 