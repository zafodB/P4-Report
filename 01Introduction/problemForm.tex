\subsection{Problem Formulation}

Smart cars utilise many different kind of communication technologies and is currently a very popular topic in today's world. Communication technology is a very broad subject regarding smart cars, to narrow and make our project more specific and manageable we use our problem formulation to stay on track.\par
We decided to focus around one functionality of smart vehicles, the platooning. Since the project is being done in Denmark, we decided to base the project in Europe. This also helps narrow down whom the stakeholders for us would be better to get in contact with.
Since this is a communication technology project, we will not be considering how law would affect the project, since it would require substantial research. The main research question thus is:
% 
\begin{itemize}
    \item \textbf{How to enable platooning of heavy duty vehicles on highways, using communication technology, in Europe?}
\end{itemize}
% 
We will investigate this subject from the following angles:
\begin{itemize}[nolistsep,noitemsep]
    \item Is the wireless technology ready to enable platooning?
    \item Who are the stakeholders in the platooning scenario?
    \item What requirements do these stakeholders have on the technology used?
\end{itemize}
% 
\paragraph{After notes}\par
Authors keep in mind that in the time of writing (Spring 2017) the platooning concept may not be able to be fully implemented, because of limited infrastructure/smart cars and current laws. The technology, equipment of cars and laws are all prone to change in upcoming years, which will render platooning able to be widly implemented in forseeable future.