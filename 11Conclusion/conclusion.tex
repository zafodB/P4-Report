\section{Conclusion}\label{sec:conclusion}
\par
%
To solve a problem of congestion, safety and fuel consumption different technologies and models have been and are developed. Platooning is one of them, which by V2V and V2I communication and additional hardware (such as radar, sensors, etc.) enable vehicle to drive at close distances autonomously by following the leading vehicle with a driver. We liked this idea a lot and therefore we decided to dig into it and try to find some solution for platooning.\par
%
The aim of ours was to research platooning and propose a possible solution for it, which would satisfy the requirements proposed in Requirement specification chapter. These requirements were based on Stakeholder analysis, secondary data and interviews. We cannot say if they were successfully met because we have not been developing whole system for platooning, we are only suggesting a possible solution for it which should be first implemented and then tested. However, the technology chosen in Analysis chapter was chosen in a way that it can safely meet requirements.\par
%
At the beginning of research about technology and collection of secondary data, we contacted several of our possible stakeholders (Scania, Car2car communication consortium, Danish road authority, Danish transport organisation for drivers) by email, kindly asking them for an interview. Unfortunately, we only got a response from Danish transport organisation for drivers from which we got a contact to a truck driver that has been testing platooning. We did an interview with him and findings were interpreted in Empirical data chapter.\par
%
Meanwhile the research about platooning was ongoing where we tried to investigate projects and researches that has been already done. We were surprised that platooning is not just something from this era, but the idea dates back to 1960s. The most interesting and relevant project such as SARTRE, Companion or SAFESPOT, were picked and read through deeply to get some inspiration, insights and secondary data. These and a few more were then processed in the Literature review chapter.\par
%
Reading and analysing existing project was not enough however. In Technology chapter, various technologies were analysed in order to answer the problem formulation better. Therefore, more research had to be conducted, in the field of technology. The V2V, V2I and VANET were studied in order to understand these principles better. This helped us with a right choice of technology. To find the right technology for platooning existing and emerging technologies such as 802.11p, ITS-G5, WAVE or ARIB were analysed with their potential use and benefits. \par
%
Having all the necessary areas researched giving us a lot of information, there is only one last thing needed and it is to analyze them and propose a solition. This is done in Analysis chapter. There our motivation is, explaining why freight companies should be interested in platooning and what the motivation for them may be. Then \par
%