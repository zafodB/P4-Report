\section{Conclusion}\label{sec:conclusion}
%
To solve a problem of congestion, safety and fuel consumption different technologies and models have been and are being developed. Platooning is one of them, which by making use of V2V and V2I communication and additional infrastructure (such as radars, sensors, etc.) enables vehicles to drive at close distances autonomously by following the leading vehicle with a driver. We liked this idea and therefore we decided to dig into it and try to find some solution for platooning.\par
%
Our aim was to research platooning and propose a possible solution for it, which would satisfy the requirements proposed in \hyperref[sec:requirements]{\textit{Requirement specification} chapter}. These requirements were based on \hyperref[sec:stakeholders]{\textit{Stakeholder analysis}}, secondary data and interviews. Not all requirements were successfully met because we have not been developing whole system for platooning, we are only suggesting a possible solution for ad hoc platooning communication. It should also be tested theoretically and practically. However, the technologies chosen in \hyperref[sec:analysis]{\textit{Analysis} chapter} were chosen in a way that they all can safely meet requirements.\par
%
At the beginning of research about technology and collection of secondary data, we contacted several of our possible stakeholders (Scania, Car2car communication consortium, Danish road authority, Danish transport organisation for drivers), kindly asking them for an interview. Unfortunately, we only got one response - from Danish transport organisation for drivers from which we got a contact - truck test driver who has been testing platooning. We did an interview with him and findings were interpreted in \hyperref[sec:data]{\textit{Empirical data} chapter}.\par
%
Meanwhile the research about platooning was ongoing where we tried to investigate projects and researches that has been already done. We were surprised that platooning is not just something from this era, but the idea dates back to 1960s. The most interesting and relevant project such as SARTRE \cite{Chan2012ProjectSARTRE}, Companion \cite{2016CompanionProject} or SAFESPOT \cite{Safespot}, were picked and analysed thoroughly to get some inspiration, insights and secondary data. These and a few more were then processed in the Literature review chapter.\par
%
Reading and analysing existing project was not enough to gather all needed data about the topic. In \hyperref[sec:technology]{\textit{Technology} chapter}, various technologies were analysed in order to answer the problem formulation better. Therefore, more research had to be conducted, in the field of technology. The V2V, V2I and VANET were studied in order to understand these principles better. To find the right technology for platooning existing and emerging technologies such as 802.11p, ITS-G5, WAVE or ARIB were analysed with their potential use and benefits. \par
%
Having all the necessary areas researched giving us a lot of information, there was only one last thing needed - analyse material and propose a solution. This is done in Analysis chapter. There motivation section explains why freight companies should be interested in platooning and what the motivation for them may be. Then an architecture of the overall system is proposed. A few architectures are discussed there with their strengths and weaknesses presented. Technology section in Analysis shows possible technologies fulfilling our requirements. We found out that all of the researched technologies does that and therefore can be used for platooning, though each of them has their pros and cons. The last part of Analysis shows advantages of autonomous systems which are in favour in platooning overally while attitude of people is not. \par
%