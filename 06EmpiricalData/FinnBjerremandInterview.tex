\subsubsection{Interview with Finn Bjerremand}

Early in the process we have contacted Morten Lindbo, press officer of DTL (Danish Transport and Logistics association). He responded with contacts of their technical consultant and test driver - Finn Bjerremand.\par
% 
Interview on 27th of April was taken via Skype call since we were located in different cities.\par
% 
Finn took part in a research project about truck platooning as a test driver. Trucks in platoon were travelling between two cities in Denmark, distance of around 300km.\par
% 
SCANIA trucks were equipped with ACC (Adaptive Cruise Control) that means it was not fully automated platoon. Finn also noticed that experience of driving with ACC in short distance (9-12 meters) platoon was very similar to regular driving. And because truck drivers usually keep very short distance between each other, Finn thinks that truck platooning will make big improvement for safety on the roads. He was excited about user experience for the drivers as platooning will increase comfort and usability.\par
% 
We introduced Finn with our research problem, and asked about his knowledge about forming ad hoc platoons while being on the road. He saw one main problem comparing to platoons that have set route - "who will pay for who?". Finn thinks it will be difficult to make ad hoc platooning beneficial for everyone, as leading car uses much more fuel than others.\par
% 
Platoon in the experiment did not use any V2V communication and was dependant on ACC for longitudinal automation and drivers for steering the wheel. Finn is also aware of fully automated experiments made by VOLVO or SCANIA, he is excited about driving experience where you don't have to do that much and can even take your breaks without stopping. Although Finn thinks that it will take long time until fully automated platooning will be possible not only on highways.\par
% 
It was relaxed and very informal interview. We greatly appreciate time from Finn Bjerremand. Interview has provided us with useful remarks about driving experience in a platoon and some concerns regarding ad hoc platooning.