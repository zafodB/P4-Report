\subsubsection{Monetised benefits of platooning}
% 
As shown in the \hyperref[sec:fuel-savings]{previous section}, platooning can save money by lower fuel consumption. But there are higher initial costs and truck will not be platooning 100\% of the time. There comes a question then, whether higher initial costs can be returned over the lifetime of truck and how much can be actually saved. In Companion project, an estimate of the money return has been done, in their deliverable 8.4 \cite{Dr.Hanelt2016CooperativeResults}. To make such a calculation, the estimate price of fuel is needed for years ahead as savings are largely dependent on the price of fuel. Therefore, a forecast for next years had to be done from which the prices will be taken. Table \ref{tab:diesel-costs} shows estimated diesel costs.
% 
\begin{table}[p]
    \centering
    \begin{tabular}{*{10}{|l}| }
        \hline
        \textbf{Year} & 2017 & 2018 & 2019 & 2020 & 2021 & 2022 & 2023 & 2024 & 2025 \\
        \hline
        \textbf{Diesel Price} & 1.18\euro & 1.18\euro & 1.18\euro & 1.20\euro & 1.22\euro & 1.23\euro & 1.25\euro & 1.27\euro & 1.29\euro\\
        \hline
    \end{tabular}
    \caption{Estimated cost of diesel. Taken from \cite[p. 34]{Dr.Hanelt2016CooperativeResults}.}
    \label{tab:diesel-costs}
\end{table}
% 
Other data has been taken also from Companion project deliverable: \textquote[\cite{Dr.Hanelt2016CooperativeResults}]{\textit{The mean age of trucks in Germany is 7.5 years, with 6.2 as the median. Hence, an average use period of seven years per truck is assumed. Relying on previous calculations, a yearly distance of 150,000 km per truck and a fuel consumption of 35 litres per 100 km was assumed. When driving in a platoon of three trucks at a speed of 70 to 80 km/h and a distance of 10 to 20 m between the trucks, the fuel saving potential per truck is 4.99\% on average.}}
Having almost all necessary variables to compute possible investment return, it is needed to choose the commerce year of platooning because of fuel prices. Year 2019 was chosen. Then the last very important thing needs to be set and that is the amount of time of which a truck will be platooning, because it will not be doing so always. Four different platooning rates has been chosen: 90\%, 70\%, 50\% and 30\%.\par
% 
\begin{table}[p]
    \centering
    \begin{tabular}{*{3}{|l}|}
        \hline
        \textbf{Platooning rate} & \textbf{Accumulated} & \textbf{Average (p.a.)}\\
        \hline
        \textbf{90\%} & 13,711.72\euro & 1,958.82\euro\\
        \hline
        \textbf{70\%} & 9,186.90\euro & 1,312.41\euro\\
        \hline
        \textbf{50\%} & 4,662.07\euro & 666.01\euro\\
        \hline
        \textbf{30\%} & 137.24\euro & 19.61\euro\\
        \hline
    \end{tabular}
    \caption{Money saved over the lifespan of truck with different platooning rate. Accumulated amount is already lower by initial costs (driving license, truck technology, annual fee, etc.). Accumulated amount is pure saving. Taken from \cite[p.34]{Dr.Hanelt2016CooperativeResults}}
    \label{tab:truck-lifespan}
\end{table}
% 
From the table \ref{tab:truck-lifespan}, it can be seen that even with a low platooning rate of 30\% the higher initial costs will be returned and moreover some small amount of money potentially earned. It is assumed that truck will be platooning more and therefore, this technology should be saving costs to company beside other benefits.