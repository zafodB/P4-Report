\subsubsection{\acrshort{wave}}
% 
\begin{center}
\textquote[\cite{VehicularTechnologySociety2014IEEEArchitecture}]{\emph{\acrshort{wave} protocols are designed to allow applications to exchange data in a consistent, interoperable, and timely manner.}}
\end{center}\par
% 
\acrfull{wave} is a set of comprising multiple standards, that aims for enabling \acrshort{V2V} and \acrshort{V2I} connectivity. Group of \acrshort{wave} standards includes extensions to the \acrshort{MAC} layer of 802.11 and several standards from the 1609 family. These standards specify the means of wireless communication on the four OSI model layers (physical, data link, network and transport layer). Furthermore, two standards specify the 'vertical' properties of the communication, such as security and remote management. Figure \ref{fig:wave-fam} shows part of the \acrshort{wave} set in detail, while Table \ref{tab:wave-stds} lists \acrshort{wave} standards and their use. The \acrshort{wave} family specifies certain frequency allocations and channels for the use in the US. While the frequencies could be adjusted for the use in other countries, standard bodies in Europe have developed own standard, thus we will skip certain details of how \acrshort{wave} works. European ITS-G5 is discussed \hyperref[sec:ITS-G5]{in the following section}.\par
% 
\begin{figure}
    \centering
    \includegraphics[width=.95\textwidth]{wave-fam}
    \caption{The standards of the \acrshort{wave} set shown together with the elements of the OSI model. Taken from \cite{VehicularTechnologySociety2014IEEEArchitecture}}
    \label{fig:wave-fam}
\end{figure}

\begin{sidewaystable}
    \centering
    \begin{tabular}{|p{5cm}|p{5cm}|p{9cm}|}
        \hline
         \textbf{IEEE Standard} & \textbf{Area} & \textbf{Features} \\
         \hline
         IEEE Std 1609.4-2010 & Extensions to IEEE 802.11 MAC layer. & 
         \begin{itemize}[nolistsep,noitemsep, topsep=0pt]
             \item Channel timing
             \item MAC addressing \& pseudo-anonymity
         \end{itemize} \\ \hline
        %  
         IEEE Std 1609.3-2010 & Networking Services & 
         \begin{itemize}[nolistsep,noitemsep, topsep=0pt]
             \item \acrshort{wave} service advertisement and channel scheduling
             \item \acrshort{wave} Short message protocol
         \end{itemize} \\ \hline
        %  
         IEEE Std 1609.2-2013 & Security Services for Applications and Management Messages &
         \begin{itemize}[nolistsep,noitemsep, topsep=0pt]
             \item Security for \acrshort{wave} Service Advertisements and \acrshort{wave} Short Messages
             \item Additional security services
         \end{itemize} \\ \hline
        %  
        IEEE Std 1609.11-2010 & Over-the-Air Electronic Payment Data Exchange Protocol for ITS & ISO-compliant payment protocol \\ \hline \hline
        %  
        \textbf{IEEE Project} & \textbf{Area} & \textbf{Features} \\ \hline
        % 
        IEEE P1609.6  & Remote Management Services & Over the air management and aliasing. \\ \hline
        % 
        IEEE P1609.5  & Communication Manager & Network management \\ \hline
    \end{tabular}
    \caption{Set of \acrshort{wave} standards and their respective areas. From \cite{VehicularTechnologySociety2014IEEEArchitecture}}.
    \label{tab:wave-stds}
\end{sidewaystable}
% 
As specified in \cite{VehicularTechnologySociety2014IEEEArchitecture}, \acrshort{wave} standards do not distinguish between types of devices connected. Instead, they are robust enough to accommodate communication to/from On-board Units (OBU), Road-side Units (RSU), together with portable units (e.g. smartphones) and pedestrian units (e.g. roadside workers). 
% 
\subsubsection*{Extensions of 802.11 MAC layer} 
The 802.11 standard and in particular its amendment - 802.11p, can be seen as the roots for the \acrshort{wave} family. 802.11p specifies the \acrshort{PHY} and \acrshort{MAC} layers of wireless communication suitable for vehicular environments. The newest revision of 1609.4 standard - \acrshort{IEEE} Std 1609.4-2016 specifies some additional features of the \acrshort{MAC} layer. It operates on frequency band 5.850GHz to 5.925GHz. While 0.005GHz at the lower edge is kept in reserve, the rest of the band is divided into 7 channels. These are further divided into \acrfull{CCH} and \acrfull{SCH}. Figure \ref{fig:wave-channels} describes channel allocation in detail.\par
% 
\begin{figure}[ht]
    \centering
    \includegraphics[width=.95\textwidth]{wave-channels}
    \caption{Figure showing channel frequencies and the use of the channels. Note that channels 174 and 176 and channels 180 and 182 can be merged to create channels 175 and 181 respectively. Taken from \cite[p. 20]{VehicularTechnologySociety2014IEEEArchitecture} (edited).}
    \label{fig:wave-channels}
\end{figure}
% 
The \acrshort{CCH} is reserved for system management messages and only allows communication via \acrfull{WSMP}. On \acrshort{SCH}, both \acrshort{IP} traffic (using IPv6 addresses) and \acrshort{WSMP} is possible. \acrshort{WSMP} is a protocol that sends \acrfull{WSM}, which are designed to consume minimal channel capacity. \acrshort{WSMP} allows transmitter device to specify physical characteristics of the transmission, such as transmitter power and channel used. The transmitter needs to provide MAC address of the receiver, however group MAC addresses are allowed. Furthermore, transmitter needs to provide a \acrfull{PSID}\footnotemark. The receiver can decide, based on the \acrshort{PSID} of received \acrshort{WSM}, if the message is of interest or should be discarded.\par
% 
\footnotetext{\textquote[\cite{20161609.12-2016Allocations}]{\acrshort{PSID} is an integer with a value from 0 to 270 549 119. [...] Each allocated \acrshort{PSID} value is associated with an organisation that is authorised to describe the use of that \acrshort{PSID}.}}
% 
A performance study comparing \acrshort{ETSI} ITS G5 and \acrshort{wave} \cite{Eckhoff2013AWAVE} suggests, that \acrshort{wave} is suitable for transmitting periodical Cooperative Awareness Messages. Those are likely the messages that would be used in platooning scenario. It may suffer from lower delivery rate and higher end-to-end delay in scenarios with high node densities and high penetration. This, however is the same for0\acrshort{ETSI} ITS G5, therefore we do not see these results as a major drawback of \acrshort{wave}. Overall, we can conclude that \acrshort{wave} is ready to facilitate platooning. After all - \emph{Vehicle communication for collision avoidance}, which shares some of the characteristics with platooning (like Forward collision warning, Longitudinal collision risk warning) and would likely need similar type of communication architecture as platooning, is a representative use case in \acrshort{IEEE} Guide for \acrshort{wave} architecture (\cite{VehicularTechnologySociety2014IEEEArchitecture}).