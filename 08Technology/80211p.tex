\subsubsection{802.11p} \label{sec:802.11p}
% 
802.11p is an amendment to standard 802.11 by \acrshort{IEEE}. This amendment was developed between 2005 and 2009 and was approved and published in 2010 as \enquote{Amendment 6: Wireless Access in Vehicular Environments}. It was included in 2012 revision of 802.11 and onwards.
802.11p is designed for \acrshort{V2V} and vehicle-to-infrastructure \acrshort{V2I} communication. This communication can often exist for short period of time only, therefore some of the properties (such as authentication) defined by original 802.11 standards are omitted. The 802.11p has served as base for the ITS-G5 standard (described in next section).\par
% 
% 
While 802.11b only introduced some changes in the \acrshort{PHY} layer to accommodate higher data rates in 2.4 GHz band, 802.11p introduces more significant changes in both \acrshort{PHY} and \acrshort{MAC} layers.
The \acrshort{PHY} characteristics of 802.11p are summed up in table \ref{table:11pPHY}.
% 
\begin{table}[h]
\centering
\begin{tabular}{|l|r|}
    \hline
    \rowcolor{lightgray} Specification & 802.11p \\
    \hline
    Data rate & 3, 4, 5, 6, 9, 12, 18, 24, 27 Mbps \\
    \hline
    Modulation & \acrshort{OFDM} \\
    \hline
    Coding rate & 1/2, 2/3, 3/4 \\
    \hline
    Number of subcarriers & 52 \\
    \hline
    \acrshort{OFDM} symbol duration & 8 $\upmu$s \\
    \hline
    Guard Time & 1.6 $\upmu$s \\
    \hline
    Preamble duration & 32 $\upmu$s \\
    \hline
    Subcarrier Spacing & 0.15625 MHz \\
    \hline
\end{tabular}
\caption{Specifications of 802.11p \acrshort{PHY} layer. Source: \cite{Abdelgader2014TheChallenges}}
\label{table:11pPHY}
\end{table}
% 
Some of the specifications are the same as for amendment 802.11a, to which 802.11p is often compared. The \acrshort{PHY} layer of 802.11p uses \acrshort{OFDM} modulation with 52 carriers.
They lie between 5,850–5,925 GHz (for the US) or 5,875–5,905 GHz (for Europe), which are frequencies licensed for vehicular use. The maximum range is 1000m in open space and maximum bandwidth is 27 Mbps \cite{Abdelgader2014TheChallenges}.
Out of those 52 carriers, 4 (\emph{pilot carriers}) are always transmitting the same pattern. This is used to correct possible phase and frequency offset on the receiver side (that may be present for example due to Doppler effect\footnotemark). The remaining 48 channels can be modulated using \acrshort{BPSK}, \acrshort{QPSK}, 16-\acrshort{QAM} or 64-\acrshort{QAM} modulation.\par
% 
\footnotetext{\textquote[Taken from: \url{https://en.wikipedia.org/wiki/Doppler_effect}, edited.]{\textit{The Doppler effect is the change in frequency or wavelength of a wave, for an observer moving relative to its source.}}}
% 
Compared to 802.11a, it has increased the guard time, just as well as symbol duration. These changes were made in order to account the problems arising from the high mobility scenarios (such as Doppler's effect).\par
% 
% \footnotetext{\url{http://www.rfwireless-world.com/Terminology/WLAN-802-11a-versus-802-11p.html}}
% 
The \acrshort{MAC} layer of 802.11p makes use of \emph{\acrfull{EDCA}}. This algorithm improves the DCF specified earlier by dividing traffic into 4 categories (so-called \emph{Access Classes - \acrshort{AC}s}):
% 
\begin{itemize}[noitemsep]
    \item background traffic (BK)
    \item best-effort traffic (BE)
    \item video traffic (VI)
    \item voice traffic (VO)
\end{itemize}
% 
Similarly to what was described in \hyperref[sec:802.11b]{802.11b}, \acrshort{CW} principle is used. The CVmax and CVmin values are specified in the standard, however, they vary for different \acrshort{AC}s. Furthermore, another parameter - \acrfull{AIFS} is defined for each of the \acrshort{AC}s. By combining \acrshort{CW} and \acrshort{AIFS} it is ensured, that more latency-sensitive content (such as voice call) will be sent sooner than not-latency-sensitive content (such as weather news).\par
% 
It has been argued \cite{Bilstrup2008EvaluationCommunication}, that vehicular use does not need different classes of content, and instead, a \emph{worst-case scenario channel access time} should be introduced. Simulations have shown, that 802.11p's \acrshort{DCF} (based on CSMA) is not very suitable for periodic broadcast messages (which are likely to be transmitted in an enhanced platooning scenario, where other vehicles need to be aware of the platoon's state.
% 
The main challenges in vehicular environment, often caused by high mobility of nodes, include signal fading, packet collisions and signal interference. The comparison of performance with 802.11b has been presented in previous section. Another possible solution for V2V communication is Japanese standard \acrshort{arib} T109 (described in following section). Performance-wise, \cite{Heinovski2016PerformanceSTD-T109} concludes, that \acrshort{arib} is more efficient in non-line of sight scenarios, which are often a case in urban environments. On the other hand, since \acrshort{arib} uses \acrshort{TDMA} (as opposed to CSMA in 802.11p), the transmission may be more delayed, as the node has to wait for next available time slot. In particular scenarios, 802.11p may be a better solution due to lower latency.\par
% 
Journal article \cite{HameedMir2014LTEEvaluation} compares 802.11p with \acrshort{LTE} network. \acrshort{LTE} is not considered in our report, however the paper brings in interesting insights. 802.11p is considered infrastructure-less standard, while \acrshort{LTE} requires significant infrastructure to operate. Though, simulations have shown, that \acrshort{LTE} performs better for most of the V2V related use-cases, due to better scheduling and channel access control.\par
\pagebreak