\section{Methodology}\label{sec:methodology}
% 
In this chapter, we will describe the methods used to find information and get a better understanding of the underlying problems, to be able to provide a potential solution to the problem formulation. This was done by researching similar projects and through their findings we learned what their problems were and how they managed to overcome them. Knowing how some of the parts was put together we were able to make a stakeholder analysis to figure out whom to interview and get an even better understanding of the whole system. Based on this information we created use cases and requirements. Knowing the requirements, we were able to look deeper into specific technologies and with some Empirical data, we could create a potential solution for the project. 
% 
\subsection{Similar projects/Literature review}
First we went through project that specifically had something to do with platooning, and how they were connected, with projects such as Sartre \cite{Chan2012ProjectSARTRE} and Companion \cite{2016CompanionProject}. But to get a better understanding on how the communication behind it works, we needed to branch out to projects, less related but still related regarding to communication technology like SAFESPOT \cite{Safespot} and number of other smaller journal articles, reports and conference proceedings regarding technologies. This gave us a better understanding on what was required for a system like platooning to work.
% 
\subsection{Stakeholder analysis}
When creating a system, many parties are involved both directly and indirectly. To get a better overview of whom gets affected and whom you should listen and take advice from a stakeholder analysis is necessary. This was used to conduct interview with the parties that the analysis suggested. 
% 
\subsection{Interview}
Using the stakeholder analysis, we could figure out who to get in contact with, to get a better understanding of the problems from their point of view. We emailed Car 2 Car Communication Consortium, Scania, Vejdirektoratet (Danish road authority) and Dansk Transport og Logistic (Danish transport organisation for drivers) asking for a representative we could interview.
Unfortunately, only DTL responded and accepted our interview, but since the representative was located in Jutland, we decided to conduct a Skype interview instead. The interview was conducted in a semi-structured way, with basic questions to make sure that the interview was going forward, but leaving the interviewer able to make follow-up questions based on the answers from the interviewee. This showed that some of the problems we expected to happen through research was reconfirmed from the interviewee, but there was also new information and problems we had not taken into consideration.
% 
\subsection{Requirement specification/use case}
After using methods mentioned above and gathering all the results from stakeholders analysis, primary and secondary interviews/surveys and state of the art research we realised that concept of platooning has a lot of of requirements, coming form different angles of the system and different stakeholders. To narrow down all possible solutions we came up with few crucial \emph{use cases}. These uses cases helped us to pick requirements that are relevant to ad hoc platooning. In Solution proposition and Conclusion we will evaluate if requirements for this project can be met and solutions for implementation are found.
% 
\subsection{Technology Research}
The communication technologies we focused mostly on were the standards, ITS-G5 (Europe), WAVE (USA) and ARIB-T109 (Japan), which are being used for V2V communication and platooning as we speak. Going through the lower layers we understood that both ITS-G5 and WAVE are using 802.11 as their fundamentals, letting us to leave WAVE out of the picture and see the difference in ITS-G5 and WAVE. This gave us a deep understanding on how the data communication between the communication devices are delivered and how to use this in our advantage.

