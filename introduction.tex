
\section{Introduction}


Travelling by car is part of everyday life of millions of people. Over the past decades, the number of vehicles on the roads has only been increasing. This brings up several challenges. How to fight the increasing pollution, to which road traffic contributes to? What to do, when there is so many cars, that they can not fit on existing roads? Some of these problems might be addressed by the concept of \emph{platooning}.\par

The term \emph{platooning} has been popularised by the project SARTRE, conducted between 2009 and 2012, which investigated, how vehicles can drive in groups, closely following each other. There are several benefits of this concept, which include fuel savings (due to drafting\footnotemark[1]), better utilisation of the road (as vehicles in platoon may drive closer to each other), increased safety and comfort of the users, since the drivers of vehicles travelling in the platoon (except the lead vehicle driver) do not need to focus on the road and may rest or work.\par


The downside of platooning are increased responsibility of the lead vehicle driver (which may require them to take additional training). Furthermore, the drivers in following vehicles may not be completely aware of road conditions at all times and can fail to take over the control of the vehicle, in case of technological errors or other unforeseen circumstances. Another issue to be addressed is how to motivate the drivers to become the lead vehicle of the platoon.

In order to achieve implementation of this concept in practice, there is number of crucial aspects to be considered, such as:
\begin{itemize}[noitemsep]
    % \setlength\itemsep{0pt}
    \item \emph{Use-cases of the system} (e.g. How is the system used? What happens when...?)
    \item \emph{Technology} (e.g. How do the vehicles communicate with each other?)
    \item \emph{Road safety} (e.g. How far apart do the vehicles need to be?)
    \item \emph{Law} (e.g. Does the law permit semi-/autonomous vehicles?)
\end{itemize} \par

There are indications, that there is research being conducted in this area, but this concept is not discussed very often in public space. Similarly, no major break-through have been announced lately that would increase the discussion about this.\par

Together with our internal motivation, which includes personal interest in the topic, relevance of the topic, in regards to both our education and on-going development of intelligent cars, and availability of resources, we decided to investigate the area further. While we will not cover all of the aspects mentioned above, there is several areas, which we pinpointed and will be researching further.\par

We will look into the concept from the IT perspective. We will come up with use cases of potential system. We will contact representatives of some of the stakeholders and will seek to carry out interviews with them, to expand our understanding of the problem. In the following part of the report, we will investigate existing technologies, standards and projects, that had to do with platooning. Namely we will look into \emph{V2V communication, ITS-G5, 802.11p} standard and other technologies and systems we may discover. Furthermore, we will briefly investigate the possibilities of implementing the concept into practice, exploring some of the economical aspects.

\footnotetext[1] {Drafting is a technique where two vehicles align, utilising the slipstream of the lead vehicle and thus reducing the air resistance of following vehicles.}

