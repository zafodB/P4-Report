\section{Requirement Specification}\label{sec:requirements}
%
In order to satisfy the stakeholders in any technological project, their requirements need to be met. The stakeholders' requirements, however, may sometimes be incomplete, contradictory or unclear. To avoid confusion, requirement specification is often conducted to clear any possible misunderstandings and present a unambiguous set of requirements a system should fulfil. In previous chapter we have identified who the stakeholders are and what their role in the system is. The scope of this report does not allow to analyse all of the requirements possible for the platooning system. Rather we will only focus on technology-related ones. We will list some of the requirements that may influence the decision made while choosing one particular technology. To assist eliciting those requirements, several use cases are presented that represent the most common actions in the platooning scenario.
%
\subsection{Use cases for ad hoc platoon}
% 
There are many different use cases in platooning. It is a very complex and large network made of many moving and fixed parts. It involves V2X communication, where messages are sent and received forth and back at very intense speeds and frequency.\par
%  
This project is mostly focused on technical side of ad hoc platoon implementation and we will only touch upon use cases that are related for this research.
% 
\subsubsection{Find/initiate platoon}
% 
This is initial use case to start using platoon service on the road. Driver using some kind of HMI implementation on his vehicle must be able to detect existing platoon to join or find other drivers that are willing to form a platoon. Long distance communication is required, for driver to be able to plan (slow down, speed up, adjust the route) his entry to platoon. To implement this use case - V2I communication could be used for long-distance implementation and extra functionality, although it must be possible to find or initiate platoon only using V2V. We will discuss technology limitations of V2V and V2I in Empirical data and Technology sections.
% 
\paragraph{Actors of the use case}
\begin{itemize}[noitemsep]
    \item Primary actor - a driver or a vehicle (in a fully automated platoon) which is willing to join/form platoon.
    \item Back-end service - a crucial part for long distance ad hoc platoon planning. It could serve for both business (logistics routing) and personal needs. It should be implemented as a remote server, that would be reached over long distance communication. This back-end solution will keep track of all existing platoons and vehicles that are looking for a platoon. When match is found it will create a route for the user to successfully meet and join the platoon or even several platoons throughout whole distance.
    \item Any other moving or fixed unit on the road (e.g. other cars acting as relays to pass information about existing platoon nearby)
\end{itemize}
% 
\paragraph{Pre-conditions}
\begin{itemize}[noitemsep]
    \item Vehicle that is suitable for platooning - equipped with technologies for V2X communication and fully/semi automated driving.
    \item Access to Internet is also critical for ad hoc platooning, although constant connection might not be necessary, for example when platoon is found and route to it is transferred to vehicle's navigation system, connection could be terminated.
\end{itemize}
% 
\paragraph{Process}
\begin{itemize}[noitemsep]
    \item With help of platoon back-end system user sends a request to find and join platoon. Request should contain details about the trip, like destination, desired travel time, stops and etc. If no platoons are available for desired trip, system would suggest to form a new platoon.
    \item Platoon service could work as spontaneous on-the-go solution, when user decides to join platoon as soon as possible. Or plan ahead option when user submits request for a platoon some time ahead, could be several days or even weeks ahead, and possibly used more by logistic companies or lone truck drivers.
    \item After available platoons are found most important details (route, number of cars and etc) are sent to the user. User can accept or decline the offer.
    \item If user has continuous Internet connection he/she could see live status and location of the platoon, otherwise user would just follow the route, received from platoon back-end. When cars are in the range of direct or relayed V2V communication, status of the platoon would be updated.
\end{itemize}
% 
There are concerns for Internet access implementation in moving vehicles and research is still in ongoing, although even periodical access should be enough for sufficient platoon organisation.\par
% 
\subsubsection{Join/leave moving platoon}
% 
Another crucial use case for platooning concept. This use case is focused around short range communication between vehicles in platoon and the ones who want to join or leave it. Before a vehicle can join/leave moving platoon, all the cars around must adapt and perform necessary actions to make sure safe process. For semi or, in future, fully automated vehicle to successfully join or leave platoon, communication must be precise without any overlapping or interference.\par
% 
\paragraph{Actors of the use case}
\begin{itemize}[noitemsep]
    \item Primary actor - vehicle that is approaching the platoon and ready to join or is in the platoon and wants to leave the platoon because of different routes or any other reasons.
    \item Secondary actors - vehicles around or in the platoon, as well as any available Road Side Unit. While joining/leaving the platoon, V2V or even V2I would make sure that it is safe to perform any actions on the road. For example any RSU or a leading car in the platoon could inform Primary actor that some slow down is ahead, and it is not safe to enter the platoon.
\end{itemize}
% 
\paragraph{Pre-conditions}
\begin{itemize}[noitemsep]
    \item Vehicle has approached platoon in a sufficient distance for V2V communication.
    \item Vehicle has to leave the platoon, for example because it needs to turn into another road or make an unplanned stop.
\end{itemize}
% 
\paragraph{Process}
\begin{itemize}[noitemsep]
    \item Vehicle approaching platoon establishes V2V communication with other vehicles in the platoon. It is up to system design to decide to which and how many cars needs to communicate to join the platoon. For example depending on a car type, or destination, user could be assigned to a different place in the platoon (e.g. front of the platoon, middle or end).
    \item When all conditions for safe join are met, vehicles might need to take some actions to let arriving car into the platoon.
    \item User successfully joins the platoon and stays in it until he/she needs to turn or platoon breaks up for some other reasons.
    \item When user must leave the platoon for any reason other cars will be informed about state change by short distance communication. Same as with joining platoon, any other participant or infrastructure on the road could communicate to the platoon, to make any actions as safe as possible.
    \item Cars in front and back performs necessary actions (e.g. make bigger gaps, slow down and etc.) and Primary actor leaves the platoon.
\end{itemize}
% 
Main concern for the use case is quality and speed of V2V communication. Driving and performing actions in high speed and very short distance between the vehicles might be dangerous, so there cannot be any errors or loss of critical messages during the process.
%
\subsection{Requirements}
After learning about the concept of platooning and establishing state of the art research, we quickly realised that there is a great number of requirements that need to be met for successful real-world platoon implementation. Project this big has a lot of stakeholders, who are very different - platoon gets great attention from governments, logistic companies, truck manufacturers, regular drivers and more. Every stakeholder sees platooning from different perspective and have different requirements for it. While not every user requirement is measured or possible to achieve, it must be considered and evaluated before it gets rejected. And this is only one side of the problem. We have discovered that there are loads of technical requirements as well. Technologies for platooning are still under development and usually there are multiple ways to meet technical requirement and every solution has its pros and cons.\par
Using data from research, primary and secondary interviews/surveys we have gathered and filtered the most relevant requirements following delimitations for this project. Following these requirements we well research possible technical solutions and in the end of project discuss if and how these specifications can be met.\par
%
\subsubsection{User requirements}
%
\begin{itemize}
    \item\textbf{Platoon must improve safety on the roads} - one of the main reasons for implementing platooning. We learned that humans are not very good drivers, there are thousands of accidents every year in the world, and most of them are caused by human error. In general, automated driving technologies are trying to minimise human mistakes by taking over vehicle's control or assisting drivers on the road.
    %
    \item\textbf{Platooning must lower fuel consumption} - every business is constantly trying to minimise costs and maximise income, in transportation fuel costs are the ones that can be lowered within help of platooning, as Companion project has proven with many different real and simulated tests. Research showed that every stakeholder of platooning expects to lower expenses - whether it's a company that manufactures goods and wants to transport them, or logistics company, or just a simple person using vehicle for travelling. On top of that - lowering fuel consumption means more environment friendly transportation.
    %
    \item\textbf{Form platoon as ad hoc network} - this solution is widely discussed in Companion project and is the most important requirement for our project as well. Forming or joining platoon while on the go means that vehicles doesn't have to start their trips from same location and can have different destinations too. It gives opportunity to save costs constantly and for everyone. For example personal car can join platoon of trucks and leave it whenever their routes separates. Trucks will be able to join and leave several platoons throughout their journey.
    %
    \item\textbf{Back-end system must plan long distance trips} - using remote back-end system users will form platoons and plan routes ahead. System must keep track of all platoons that are or will be on the road. This way planned platoons can be combined with users who are willing to join on the go as well as change routes, reform platoons and much more.
    %
    \item\textbf{Ad hoc platooning must be reliable} - while this is very broad requirement it is self explanatory as well. For fully or semi automated ad hoc platoon to ever be even allowed on the roads, every bit of the system must be reliable. From technical implementation to route planning, everything must be tested and confirmed.
    %
    \item\textbf{Improved driving experience} - from interviews and surveys we noticed that drivers asked about platooning, firstly think about more comfortable driving. Platoon is aiming to provide better work conditions and efficiency by taking some responsibilities from drivers. Allowing drivers take their breaks while travelling in platoon would not only improve workers satisfaction but also would save time for the companies.
\end{itemize}
%
\subsubsection{Technical requirements}
%
All the research from sections above and functional requirements leads us to general functional requirements. In this project we mostly focus on communication solutions and requirements listed below has to be met to successfully implement ad hoc platooning.
%
\begin{itemize}
    \item\textbf{V2V must use short-range communication technology} - V2V communication must be implemented with most suitable technology for the case. There are quite few standards around the world, but automated driving is not ready yet for real-world stage and we must evaluate each standards advantages and disadvantages.
    %
    \item\textbf{Connect to backend service over wireless technology} - for planning/finding platoons vehicle must communicate to remote backend service. Vehicle must have at least periodical Internet connection for successful route planning.
    %
    \item\textbf{Join/find platoon without any V2I communication} - if user cannot access Internet to use platoon back-end service, there must be possibility to detect or join platoon using only V2V communication.
    %
    \item\textbf{V2I must allow non-highway platooning} - where V2I communication is available, meaning user has Internet access and road infrastructure is implemented (e.g. traffic lights, any other RSU) platooning must be possible on non-highway roads. 
    %
    \item\textbf{System must be able suitable for different type/brand vehicles} - ad hoc platooning must be available for different brand vehicles. Logistics company might owe different vehicles, so for ad hoc platooning to expand it must be a kind of "cross-platform" solution.
    %
    \item\textbf{Critical messages must reach destination} - critical messages like crash/break warnings must arrive at right order, they cannot be lost or interfered by any other irrelevant communication.
\end{itemize}\par
%
Requirements above were gathered using primary and secondary data, as well as our own brainstorming and knowledge. These seem to be critical requirements for successful ad hoc platoon communication. Solutions or propositions for these requirements will be provided in Solution Proposition chapter.

