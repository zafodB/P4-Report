\subsection{\textit{Companion}}\label{sec:Companion}
% 
A very influential research project for platoon implementation in near future, named \emph{Companion} \cite{2016CompanionProject} has been established by SCANIA between 2013 and 2016. Project was coordinated by Magnus Adolfson and involved such automotive giant as Volkswagen.\par
% 
Project's full definition looked like this - "Cooperative dynamic formation of platoons for safe and energy-optimised goods transportation"\footnotemark and it is concentrated on research and development of mobility technologies for supervised platooning. Companion's main objective is to improve fuel efficiency and safety of transportation.%
% 
\footnotetext{\url{http://cordis.europa.eu/project/rcn/110628_en.html}, accessed 10/04/2017}%
% 
Problems that are analysed by Companion, are closely related to the ones we are trying to solve with our road train solution in this project and it seems worthwhile investigating work done by experts in the field\footnotemark.\par
% 
\footnotetext{\url{https://youtu.be/_19ui-8f8cw}, accessed 10/04/2017}
% 
Solution suggested by Companion is somewhat new comparing to other platoon projects. They want to form platoons dynamically - meaning vehicle could join and leave platoon whenever it is suitable for that particular vehicle. Participant doesn't have to stay in the platoon for the whole trip, every vehicle can plan different starting and destination locations. Cars would use V2V communication to be able to find nearest platoons and then when in platoon mode, vehicles would communicate constantly to steer, brake and accelerate following vehicles (FV). It is a solution that is not dependant on road infrastructure. Also worth mentioning that project was mostly focused on heavy-duty vehicles.\par
% 
Scope for the project Companion - development and prototype implementation of \emph{Off-Board system for Platoon Coordination} and \emph{On-Board System for Coordinated Vehicle Platooning}.\par
% 
\emph{Off-Board system} will consist route calculation and optimisation engines and off-board HMI. The system would be controlled by remote dispatchers and is most suitable for planning long distance goods transportation.\par
% 
For developing \emph{On-Board V2V system} interviews with drivers took place. They covered drivers opinions on vehicle HMIs and platooning in general. Results showed that drivers are positive about new technologies and idea of platooning. Although they would have to build trust of the system for short-distance driving.\par
% 
It is fair to conclude that Companion project was successful in sense that it proved platoon to be high potential and near future technology. Physical and practical tests showed that it could lower fuel consumption up to 12.6\%. Various simulations and driver interviews showed that platooning could drastically improve safety and congestion problems on the roads. Results of various simulations and tests established during Companion project will be analysed more in depth in empirical data section in this project.
% 