\subsection{\textit{A review of Truck Platooning Projects for Energy Savings}}

This paper \cite{Tsugawa2016ASavings}, published in \emph{Transactions on Intelligent Vehicles}, an IEEE journal, serves as a review of platooning projects that has been done over past decades. Also, it serves as an overview of what platooning is, why there have been projects done since 1980s, what are the objectives of automated platooning and what technologies/projects have been developed since (the most notable ones, according to authors).
This paper classifies objectives of platooning to three main categories: Energy consumption and CO2 emission, characteristics of the trucking industry and transportation capacity. However, they also realise other positive aspects of it such as safety increase, congestion reduction and comfort.\par
% 
Projects being described in the paper are - Automated Trucks in Energy ITS, Electronically Coupled Truck Platoons “KONVOI”, PATH Development of Truck Platooning and CACC Systems. Paper describes purpose of each project. It also elaborates on longitudinal and lateral control system of each project in detail. The architecture of each system was presented, showing all components of the system and their interaction with each other. For communication between trucks, V2V communication is being used and all of its components are being presented, as well as what information are sent/received and what is payload (for the projects, where these information are available).\par
% 
The last section of this paper is about impacts of platooning. There are presented findings of the previously mentioned projects. Those findings are mostly showing about fuel savings at different gaps between truck.\par
% 
This paper helped us to get better understanding of platooning as it compares several project’s main ideas, aims, technology and findings. Because of that, we were able to get a lot of secondary data for our project, without a need for going through each project separately as all necessary data are presented there.
