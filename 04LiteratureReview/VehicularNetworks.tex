\subsection{\textit{Vehicular Networks}}

% 
Vehicular Ad Hoc Networks (VANETs) are gaining significant amount of interest in studies and industry. It is believed that implementation of V2V and V2I communications will lead to success in transportation business in a near future. A report established by Caixing Shao and Supeng Leng from University of Electronic Science and Technology in China \cite{Shao2014AnalysisNetworks} analyses VANET probability in a platoon in a detailed manner.\par
% 
Their V2I solution is tightly coupled with RSU (Road Side Units) implementation on the roads. A problem we are solving in this project is mostly dependant on successful V2V communication and is not meant to be dependant on road infrastructures. With that in mind continuous V2I communication is not necessary for this project, but having some V2I (periodical Internet connection) would definitely be beneficial for safety and user experience.\par
% 
Analysis made in University of Electronic Science and Technology in China offers a intelligent solution for continuous Internet access through road side units even if the vehicle is not in the range of RSU. It would use platoon vehicles as relays to reach access to Internet. It is an idea that has potential to be implemented worldwide, so we decided to dig further.\par
% 
Since our platoon solution is not dependant on continuous V2I communication and RSUs are not being implemented in a big scale at the moment \cite{Tonguz2013CarsSolution}, we researched the core idea of using other vehicles in the platoon as relays. Keeping in mind that cars in the platoon will have V2V connection anyways. We have found an interesting report made by students of Carnegie Mellon University in Thailand \emph{Cars as Roadside Units} \cite{Tonguz2013CarsSolution}.\par
% 
Their idea is based on vehicles adopting new wireless communication, that is intentionally developed for V2X (Vehicle to everything) - DSRC (Dedicated Short Range Communication) \cite{OfficeoftheAssistantSecretaryforResearchandTechnologyIntelligentSheet}.
Report is proving that implementing RSUs widely might be too costly and inefficient, so they suggest using DSRC enabled vehicles as temporary Road Side Units on planned briefly stops during the trip. Since the car is acting as RSU for short period of time it would mostly be used to send out messages in case of accident or congestion in ROI (range of interest).